\section{Background}

This section provides the technical context necessary to understand our formal verification of TCP-AAD.

\subsection{TCP Delayed Acknowledgment}

The Transmission Control Protocol (TCP) is the dominant transport protocol for reliable data transmission on the Internet~\cite{rfc793}. To reduce network overhead, TCP receivers use \textit{delayed acknowledgment} (DACK), where acknowledgments (ACKs) for received data segments are intentionally delayed before transmission.

\subsubsection{RFC 1122 Requirements}

RFC 1122~\cite{rfc1122} specifies the standard TCP delayed acknowledgment behavior:

\begin{itemize}
    \item \textbf{Segment Rule}: ACK at least every second full-sized segment
    \item \textbf{Timeout Rule}: ACK within 500ms if no second segment arrives
    \item \textbf{Out-of-order Rule}: Send immediate ACK for out-of-order segments
\end{itemize}

These rules balance two competing goals: (1) reducing ACK overhead by acknowledging multiple segments with a single ACK, and (2) maintaining fast feedback to the sender for congestion control and loss detection.

The default Linux kernel implementation uses a fixed 500ms timeout, which works well for traditional wired networks but may be suboptimal for modern Wi-Fi networks with frame aggregation.

\subsection{IEEE 802.11 Frame Aggregation}

Modern Wi-Fi standards (802.11n/ac/ax) introduced \textit{frame aggregation} to improve efficiency~\cite{ieee80211n}. Instead of transmitting MAC frames individually, the wireless driver batches multiple frames into a single transmission:

\begin{itemize}
    \item \textbf{A-MSDU}: Aggregates multiple MSDUs (MAC Service Data Units) before MAC layer processing
    \item \textbf{A-MPDU}: Aggregates multiple MPDUs (MAC Protocol Data Units) at the MAC layer
\end{itemize}

Frame aggregation significantly reduces per-frame overhead but introduces \textit{bursty} traffic patterns. Multiple TCP segments may arrive at the receiver with very small inter-arrival times (tens of microseconds) followed by longer gaps.

\subsection{The TCP-AAD Algorithm}

TCP-AAD (Aggregation-Aware Delayed Acknowledgment)~\cite{albert_thesis} adapts the ACK timeout dynamically based on observed inter-arrival times (IAT). The key insight is that the optimal ACK timeout should be proportional to the time between segment arrivals.

\subsubsection{IAT Tracking}

TCP-AAD maintains two metrics:

\begin{itemize}
    \item $IAT_{min}$: Minimum inter-arrival time observed in the current period
    \item $IAT_{curr}$: Current inter-arrival time (time since last segment)
\end{itemize}

Small IATs (below 0.2ms) are filtered out as noise to avoid overly aggressive adaptation.

\subsubsection{Adaptive Timeout Calculation}

The adaptive timeout (ATO) is calculated using an exponential weighted moving average:

\begin{equation}
ATO = (IAT_{min} \times 0.75 + IAT_{curr} \times 0.25) \times 1.5
\end{equation}

The factor of 1.5 provides headroom to avoid premature timeouts. The ATO is bounded by a maximum value (500ms by default) to maintain RFC 1122 compliance.

\subsubsection{Periodic Reset}

To adapt to changing network conditions, $IAT_{min}$ is reset to infinity every 1 second. This allows the algorithm to track both short-term and long-term changes in arrival patterns.

\subsection{Formal Verification of Network Protocols}

Formal verification uses mathematical techniques to prove properties about systems. In the context of network protocols, verification can detect subtle bugs that are difficult to find through testing~\cite{musuvathi2004model}.

\subsubsection{Model Checking}

Model checking~\cite{clarke1999model} systematically explores all possible states of a system to verify temporal logic properties. The SPIN model checker~\cite{holzmann1997model} uses:

\begin{itemize}
    \item \textbf{Promela}: A modeling language for specifying concurrent systems
    \item \textbf{LTL}: Linear Temporal Logic for specifying properties
    \item \textbf{State Space Exploration}: Exhaustive or partial search of reachable states
\end{itemize}

\subsubsection{Challenges in Protocol Verification}

Verifying TCP presents several challenges:

\begin{enumerate}
    \item \textbf{State Space Explosion}: Large parameter spaces create billions of states
    \item \textbf{Time Modeling}: SPIN is untimed; timing behavior requires abstraction
    \item \textbf{Concurrency}: Sender, receiver, and timers execute concurrently
    \item \textbf{Property Specification}: Correctness must be formalized in LTL
\end{enumerate}

Our work addresses these challenges through careful abstraction and bounded model checking.
